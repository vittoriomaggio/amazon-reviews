\chapter{Conclusioni}
	Durante l'elaborato abbiamo eseguito un'analisi sul \textit{sentiment} prima e sulla \textit{network} poi, per rispondere ad alcune domande prefissate o per osservare determinati cambiamenti. Riassumiamo a questo punto tutte le informazioni ricavate dalle domande, e all'utilità che queste forniscono all'azienda Amazon, che abbiamo supposto essere nostro cliente. \\	
	La nostra prima analisi ha evidenziato quali fossero i cinque prodotti più recensiti all'interno delle quattro categorie prese in questione. Sulla base dei risultati ottenuti possiamo consigliare al nostro cliente un'organizzazione del sito basata proprio su queste mini liste di prodotti o ancora di investire su altri, compresi quelli esclusi dall'elenco, ma aventi caratteristiche simili ai "migliori". In aggiunta, il poter conoscere le parole più frequenti utilizzate da un utente per una critica, può permettere ad Amazon di contattare i venditori per redarguirli su alcune pecche di qualità dei prodotti o ancora di auto migliorarsi, andando a correggere tutti quelli aspetti negativi riscontrati, come la spedizione o il prezzo. Abbiamo inoltre notato l'importanza delle recensioni negative rispetto quelle positive; in quanto l'utente trova molto più utili le prime rispetto che le seconde. E' quindi necessaria una politica di continuo miglioramento da parte di entrambi. Questo aspetto viene rispecchiato anche nella valutazione espressa in stelle; Amazon, infatti, calcola le stelle di valutazione di un prodotto tramite un modello ad apprendimento automatico invece che tramite una media di dati grezzi. Il modello ad apprendimento automatico include fattori come: la data di una recensione, i voti di utilità da parte dei clienti e il fatto che una recensione provenga o meno da acquisti verificati.\\
	
	Torniamo ora alla questione prezzo, sulla quale è stata eseguita un'analisi più approfondita, per visualizzare la correlazione tra recensioni e prezzo legate a uno specifico prodotto. Lo studio ha evidenziato che tre categorie su quattro hanno una buona distribuzione di recensioni positive e negative per ogni fascia di prezzo. L'unica eccezione riguarda i libri, dove quasi tutta la totalità di recensioni negative sono concentrate sulla prima fascia. Questa diversità così evidente rispetto alle altre occorre che sia presa seriamente in esame da Amazon, poiché dimostra che il discontento è legato limitatamente a una specifica fascia. Tramite lo studio della rete abbiamo inoltre verificato che i prodotti più visti coincidono, anche se non totalmente, con quelli più recensiti. Dopo questa informazione Amazon dovrà cercare di investire i propri fondi a sostegno di questi prodotti. Nuove informazioni sono state invece appresa dalla suddivisione dei videogiochi in comunità; per ogni categoria abbiamo infatti trovato il prodotto più rilevante, quello su cui finisce più spesso l'utente durante la sua fase di ricerca. Con un risultato del genere siamo in grado di fornire ad Amazon il prodotto da sponsorizzare per ogni categoria.

