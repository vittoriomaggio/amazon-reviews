\chapter{Conclusioni}
	Durante l'elaborato abbiamo eseguito un'analisi sul \textit{sentiment} prima e sulla \textit{network} poi, per rispondere ad alcune domande prefissate o per osservare determinati cambiamenti. Riassumiamo a questo punto tutte le informazioni ricavate dalle domande, e all'utilità che queste possono portare nella vita di utenti, venditori \textit{amazon}, aziende, etc.
	
	\begin{itemize}
		\item \textbf{distribuzione del sentiment negli anni}
		Ci sono utenti attivi? 
		Varia in base a categoria? 
		A chi può essere utile?
		
		\item \textbf{intrattenimento} 
		Una volta che ho trovato i primi 5 prodotti per ogni categoria, questo a chi giova?
		
		\item \textbf{correlazione tra prezzo e sentiment} 
		il sentiment delle recensioni è legato al prezzo? 
		A chi serve saperlo?
		
		\item \textbf{utilità recensioni per utenti}
		Sono più utili le recensioni positive o negative?
		Abbiamo visto che sono iù utili quelle negative, il venditore, dovrà quindi occuparsi di reagire alle critiche andando a riparare il suo prodotto, sia se questo presenta malfunzionamenti, sia per altri motivi.
		
		\item \textbf{prodotti più popolari visti}
		Produttori possono vedere quale gioco ha più caratteristiche comuni con quelli in voga al momento e adattarsi al trend

	\end{itemize}