\chapter{Network Analysis}
	In questo capitolo inizia la seconda parte di analisi, quella relativa alal rete. L’analisi delle reti sociali (o Network Analysis) rappresenta un insieme di strumenti finalizzati
	a descrivere le principali caratteristiche di una struttura di nodi e connessioni rifacendosi alla teoria dei grafi. Un grafo è definito come un insieme di coppie ordinate: \\
	\verb|G=(V,A)|, \\
	dove con \verb|V| si indicano l'insieme di vertici e con \verb|A| l'insieme di archi.	Un grafo può essere orientato (directed) o non orientato (non-directed). Nel primo caso, i	legami che connettono i nodi hanno una direzionalità (in uscita da un nodo e in entrata in un altro nodo), mentre nel secondo la relazione non ha un orientamento definito. 
	
	Il nostro approccio alla rete ha riguardato il \textit{dataset} prodotti; in particolare abbiamo scelto di rappresentare un grafo per le categorie "libri", "dvd", "videogiochi", "musica" in cui i nodi fossero rappresentati dagli id dei prodotti e gli archi dagli attributi \verb|also_bought|, \verb|also viewed|, \verb|bought_together|.
	
	\section{title}
		