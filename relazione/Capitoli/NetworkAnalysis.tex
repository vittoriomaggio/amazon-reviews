\chapter{Network Analysis}
\ref{cap:NetworkAnalysis}
	In questo capitolo inizia la seconda parte di analisi, quella relativa alal rete. L’analisi delle reti sociali (o Network Analysis) rappresenta un insieme di strumenti finalizzati
	a descrivere le principali caratteristiche di una struttura di nodi e connessioni rifacendosi alla teoria dei grafi. Un grafo è definito come un insieme di coppie ordinate: \\
	\verb|G=(V,A)|, \\
	dove con \verb|V| si indicano l'insieme di vertici e con \verb|A| l'insieme di archi.	Un grafo può essere orientato (directed) o non orientato (non-directed). Nel primo caso, i	legami che connettono i nodi hanno una direzionalità (in uscita da un nodo e in entrata in un altro nodo), mentre nel secondo la relazione non ha un orientamento definito. 
	
	
	\section{Analisi sulla rete: videogiochi}
		??? manca Figura?????
		Il nostro approccio alla rete ha riguardato il \textit{dataset} prodotti; in particolare abbiamo scelto di rappresentare un grafo per la categoria "videogiochi", in cui i nodi fossero rappresentati dagli id dei prodotti e gli archi dall'attributo \verb|also viewed|.	A questo punto abbiamo proceduto con la rilevazione delle comunità tramite il metodo \verb|Louvian|, che sarà approfondito nella sezione seguente.
		
		\subsection{Metodo Louvain}
			Il metodo \textit{Louvain} per il rilevamento di comunità è un metodo per estrarre comunità da grandi reti. L'ispirazione per questo metodo di rilevamento della comunità è l'ottimizzazione della modularità man mano che l'algoritmo progredisce. La modularità è un valore di scala compreso tra -0,5 (clustering non modulare) e 1 (clustering completamente modulare) che misura la densità relativa dei bordi all'interno delle comunità rispetto ai bordi esterni alle comunità. L'ottimizzazione di questo valore si traduce teoricamente nel migliore raggruppamento possibile dei nodi di una determinata rete, tuttavia passare attraverso tutte le possibili iterazioni dei nodi in gruppi non è pratico, quindi vengono utilizzati algoritmi euristici. Nel metodo \textit{Louvain} di rilevamento della comunità, le prime piccole comunità vengono trovate ottimizzando localmente la modularità su tutti i nodi, quindi ogni piccola comunità viene raggruppata in un nodo e il primo passaggio viene ripetuto. Il metodo è simile al metodo precedente di Clauset, Newman e Moore [3] che collega le comunità la cui fusione produce il maggiore aumento della modularità.
		
		\section{Identificazione comunità}
			Lo scenario che si presentava era composto da quasi \verb|100| comunità; numero ancora troppo elevato per le nostre analisi. La nostra scelta è stata quindi quella di raggruppare nuovamente le comunità per il campo console. L'operazione ha portato all'identificazione delle seguenti \textit{community}
			\begin{itemize}			
				\item Altro
				\item Nintento Switch
				\item PS4
				\item Xbox one
				\item Nintento 3DS
				\item PS3
				\item PSVita
				\item Nintento Classic Mini
				\item Nintento Wii
				\item Xbox 360
			\end{itemize}
			